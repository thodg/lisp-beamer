\documentclass[8pt]{beamer}

\usepackage[frenchb]{babel}
\usepackage[T1]{fontenc}
\usepackage[utf8]{inputenc}

\usetheme{Warsaw}

\AtBeginSection[]{
  \begin{frame}
  \vfill
  \centering
  \begin{beamercolorbox}[sep=8pt,center,shadow=true,rounded=true]{title}
    \usebeamerfont{title}\insertsectionhead\par%
  \end{beamercolorbox}
  \vfill
  \end{frame}
}

\title{Rails on Lisp}
\author{Thomas de Grivel <thoxdg@gmail.com>}
\institute{http://kmx.io}
\date{\today}

\begin{document}

\begin{frame}
\titlepage
\end{frame}

\section{Intro}

\begin{frame}
  \frametitle{Common Lisp}
  Common Lisp is the programmable programming language.
\end{frame}

\begin{frame}
  \frametitle{Common Lisp}
  Standardisé en 1994 par l'ANSI \\
  \url{http://www.lispworks.com/documentation/HyperSpec/Front/}
\end{frame}

\begin{frame}
  \frametitle{Common Lisp}
  De nombreux compilateurs respectant le standard ANSI existent :
  \begin{itemize}
  \item SBCL (open-source, x86, amd64, Windows, Linux, OSX, *BSD)
  \item ABCL (open-source, jvm)
  \item Clozure CL (open-source, x86, amd64, Windows, Linux, OSX, FreeBSD)
  \item ECL (open-source, compiles to C)
  \item LispWorks (proprietary, x86, amd64, Windows, Linux, OSX, FreeBSD)
  \item Allegro CL (proprietary, x86, amd64, sparc, Windows, Linux, OSX, FreeBSD)
  \end{itemize}
\end{frame}

\begin{frame}
  \frametitle{Common Lisp}
  Lisp essays by Paul Graham \\
  \url{http://www.paulgraham.com/lisp.html}
\end{frame}

\section{Setup}

\begin{frame}[fragile]
  \frametitle{Installer SBCL}
  Ubuntu :
\begin{verbatim}
  sudo apt-get install sbcl
\end{verbatim}
  MacOS X :
\begin{verbatim}
  brew install sbcl
\end{verbatim}
\end{frame}

\begin{frame}[fragile]
  \frametitle{Installer repo}
\begin{verbatim}
mkdir -p ~/common-lisp/thodg
cd ~/common-lisp/thodg
git clone https://github.com/thodg/repo.git
cd ~/common-lisp
ln -s thodg/repo/repo.manifest
\end{verbatim}
\end{frame}

\begin{frame}[fragile]
  \frametitle{Configurer SBCL}
  Éditer \tt{\textasciitilde/.sbclrc}
\begin{verbatim}
  ;; ASDF
  (require :asdf)

  ;; repo
  (load "~/common-lisp/thodg/repo/repo")
  (repo:boot)
\end{verbatim}
\end{frame}

\begin{frame}[fragile]
  \frametitle{Lancer SBCL}
\begin{verbatim}
$ sbcl
This is SBCL 1.5.3, an implementation of ANSI Common Lisp.
More information about SBCL is available at <http://www.sbcl.org/>.

SBCL is free software, provided as is, with absolutely no warranty.
It is mostly in the public domain; some portions are provided under
BSD-style licenses.  See the CREDITS and COPYING files in the
distribution for more information.

* _
\end{verbatim}
\end{frame}

\begin{frame}[fragile]
  \frametitle{Installer Slime}
\begin{verbatim}
* (repo:install :slime)

$ /usr/bin/git -C /home/dx/common-lisp/slime clone https://github.com/slime/slime.git
Cloning into 'slime'...
\end{verbatim}
\end{frame}

\begin{frame}[fragile]
  \frametitle{Configurer emacs}
  Éditer \tt{\textasciitilde/.emacs}
\begin{verbatim}
  ;; Common Lisp
  (add-to-list 'load-path "~/common-lisp/slime/slime/")
  (require 'slime-autoloads)
  (add-to-list 'slime-contribs 'slime-fancy)
  (setq inferior-lisp-program
        "sbcl")
  (setq slime-net-coding-system
        'utf-8-unix)
\end{verbatim}
\end{frame}

\begin{frame}[fragile]
  \frametitle{Lancer emacs et slime}
\begin{verbatim}
$ emacs

M-x slime

CL-USER> _
\end{verbatim}
\end{frame}

\section{Common Lisp}

\begin{frame}[fragile]
  \frametitle{La REPL}
  REPL : {\tt read}, {\tt eval}, {\tt print loop}
\begin{verbatim}
  (loop
    ;; setup REPL vars
    ;; handle errors, interactive debugger
    (print
      (eval
        (read)))
    (force-output)) ;; flush output buffers
\end{verbatim}
\end{frame}

\begin{frame}[fragile]
  \frametitle{Les symboles}
  Un symbole est plus rapide à comparer qu'une chaîne de caractères
  (comparaison de pointeurs).
  Pour récupérer un symbole à travers {\tt eval} il faut le quoter en le préfixant d'une apostrophe.
\begin{verbatim}
  ;; SLIME
  CL-USER> 'hello-world
  
  HELLO WORLD
  CL-USER> (quote hello-world)     ; equivalent sans syntaxe
  
  HELLO WORLD
\end{verbatim}
\end{frame}

\begin{frame}[fragile]
  \frametitle{Les symboles}
  Si on ne quote pas le symbole on tombe dans le debugger interactif.
\begin{verbatim}
  ;; SLIME
  CL-USER> hello-world
  
  The variable HELLO-WORLD is unbound.
     [Condition of type UNBOUND-VARIABLE]
  
  Restarts:
   0: [CONTINUE] Retry using HELLO-WORLD.
   1: [USE-VALUE] Use specified value.
   2: [STORE-VALUE] Set specified value and use it.
   3: [RETRY] Retry SLIME REPL evaluation request.
   4: [*ABORT] Return to SLIME's top level.
   5: [ABORT] abort thread (#<THREAD "repl-thread" RUNNING {1003B91BC3}>)
  
  Backtrace:
    0: (SB-INT:SIMPLE-EVAL-IN-LEXENV HELLO-WORLD #<NULL-LEXENV>)
    1: (EVAL HELLO-WORLD)
   --more--
  
  4
  ; Evaluation aborted on #<UNBOUND-VARIABLE HELLO-WORLD {1004AF3523}>.
  CL-USER> _
\end{verbatim}
\end{frame}

\begin{frame}[fragile]
  \frametitle{Les fonctions}
  Pour définir une fonction on utilise {\tt defun}.
  Si le premier élément d'une liste (entre parenthèses) est une fonction ou un symbole nommant une fonction alors c'est un appel de fonction.
\begin{verbatim}
  ;; SLIME
  CL-USER> (defun hello-world ()
             (format t "Hello world !"))
  HELLO-WORLD
  CL-USER> (hello-world)
  Hello world !
  NIL
  CL-USER> _
\end{verbatim}
\end{frame}

\begin{frame}[fragile]
  \frametitle{Lambda}
  Une fonction anonyme est introduite par {\tt lambda}. On peut affecter une fonction anonyme à un symbole, reproduisant l'effet de {\tt defun}.
\begin{verbatim}
  ;; SLIME
  CL-USER> (setf (symbol-function 'hello-world)
                 (lambda ()
                   (format t "Hello world !")))
  
  CL-USER> (hello-world)
  Hello world !
  NIL
  CL-USER> _
\end{verbatim}
\end{frame}

\begin{frame}[fragile]
  \frametitle{Les fonctions d'ordre supérieur}
  Une fonction est une valeur comme une autre et peut être passée en paramètre d'une autre fonction. On appelle ces fonctions les fonctions d'ordre supérieur.
\begin{verbatim}
  ;; SLIME
  CL-USER> (mapcar (lambda (x) (* x x)) '(1 2 3 4 5))
  (1 4 9 16 25)
  CL-USER> (reduce #'+ '(1 2 3 4 5))
  15
  CL-USER> (reduce (function +) '(1 2 3 4 5))
  15
  CL-USER> (reduce '+ '(1 2 3 4 5))
  15
  CL-USER> _
\end{verbatim}
\end{frame}

\begin{frame}[fragile]
  \frametitle{Les macros}
  Les paramètres d'une macro ne sont pas évalués. Cela permet de faire des DSL et de la meta-programmation.
  Une macro génère du code qui est à son tour évalué.
\begin{verbatim}
  ;; SLIME
  CL-USER> (defmacro hello (arg)
             `(format nil "Hello ~A !"
                (string-capitalize ',arg)))
  HELLO
  CL-USER> (hello world)
  "Hello World !"
  CL-USER> _
\end{verbatim}
\end{frame}

\end{document}
